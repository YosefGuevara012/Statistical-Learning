% Options for packages loaded elsewhere
\PassOptionsToPackage{unicode}{hyperref}
\PassOptionsToPackage{hyphens}{url}
%
\documentclass[
]{article}
\usepackage{amsmath,amssymb}
\usepackage{lmodern}
\usepackage{ifxetex,ifluatex}
\ifnum 0\ifxetex 1\fi\ifluatex 1\fi=0 % if pdftex
  \usepackage[T1]{fontenc}
  \usepackage[utf8]{inputenc}
  \usepackage{textcomp} % provide euro and other symbols
\else % if luatex or xetex
  \usepackage{unicode-math}
  \defaultfontfeatures{Scale=MatchLowercase}
  \defaultfontfeatures[\rmfamily]{Ligatures=TeX,Scale=1}
\fi
% Use upquote if available, for straight quotes in verbatim environments
\IfFileExists{upquote.sty}{\usepackage{upquote}}{}
\IfFileExists{microtype.sty}{% use microtype if available
  \usepackage[]{microtype}
  \UseMicrotypeSet[protrusion]{basicmath} % disable protrusion for tt fonts
}{}
\makeatletter
\@ifundefined{KOMAClassName}{% if non-KOMA class
  \IfFileExists{parskip.sty}{%
    \usepackage{parskip}
  }{% else
    \setlength{\parindent}{0pt}
    \setlength{\parskip}{6pt plus 2pt minus 1pt}}
}{% if KOMA class
  \KOMAoptions{parskip=half}}
\makeatother
\usepackage{xcolor}
\IfFileExists{xurl.sty}{\usepackage{xurl}}{} % add URL line breaks if available
\IfFileExists{bookmark.sty}{\usepackage{bookmark}}{\usepackage{hyperref}}
\hypersetup{
  pdftitle={Cap 2. Exercises},
  hidelinks,
  pdfcreator={LaTeX via pandoc}}
\urlstyle{same} % disable monospaced font for URLs
\usepackage[margin=1in]{geometry}
\usepackage{color}
\usepackage{fancyvrb}
\newcommand{\VerbBar}{|}
\newcommand{\VERB}{\Verb[commandchars=\\\{\}]}
\DefineVerbatimEnvironment{Highlighting}{Verbatim}{commandchars=\\\{\}}
% Add ',fontsize=\small' for more characters per line
\usepackage{framed}
\definecolor{shadecolor}{RGB}{248,248,248}
\newenvironment{Shaded}{\begin{snugshade}}{\end{snugshade}}
\newcommand{\AlertTok}[1]{\textcolor[rgb]{0.94,0.16,0.16}{#1}}
\newcommand{\AnnotationTok}[1]{\textcolor[rgb]{0.56,0.35,0.01}{\textbf{\textit{#1}}}}
\newcommand{\AttributeTok}[1]{\textcolor[rgb]{0.77,0.63,0.00}{#1}}
\newcommand{\BaseNTok}[1]{\textcolor[rgb]{0.00,0.00,0.81}{#1}}
\newcommand{\BuiltInTok}[1]{#1}
\newcommand{\CharTok}[1]{\textcolor[rgb]{0.31,0.60,0.02}{#1}}
\newcommand{\CommentTok}[1]{\textcolor[rgb]{0.56,0.35,0.01}{\textit{#1}}}
\newcommand{\CommentVarTok}[1]{\textcolor[rgb]{0.56,0.35,0.01}{\textbf{\textit{#1}}}}
\newcommand{\ConstantTok}[1]{\textcolor[rgb]{0.00,0.00,0.00}{#1}}
\newcommand{\ControlFlowTok}[1]{\textcolor[rgb]{0.13,0.29,0.53}{\textbf{#1}}}
\newcommand{\DataTypeTok}[1]{\textcolor[rgb]{0.13,0.29,0.53}{#1}}
\newcommand{\DecValTok}[1]{\textcolor[rgb]{0.00,0.00,0.81}{#1}}
\newcommand{\DocumentationTok}[1]{\textcolor[rgb]{0.56,0.35,0.01}{\textbf{\textit{#1}}}}
\newcommand{\ErrorTok}[1]{\textcolor[rgb]{0.64,0.00,0.00}{\textbf{#1}}}
\newcommand{\ExtensionTok}[1]{#1}
\newcommand{\FloatTok}[1]{\textcolor[rgb]{0.00,0.00,0.81}{#1}}
\newcommand{\FunctionTok}[1]{\textcolor[rgb]{0.00,0.00,0.00}{#1}}
\newcommand{\ImportTok}[1]{#1}
\newcommand{\InformationTok}[1]{\textcolor[rgb]{0.56,0.35,0.01}{\textbf{\textit{#1}}}}
\newcommand{\KeywordTok}[1]{\textcolor[rgb]{0.13,0.29,0.53}{\textbf{#1}}}
\newcommand{\NormalTok}[1]{#1}
\newcommand{\OperatorTok}[1]{\textcolor[rgb]{0.81,0.36,0.00}{\textbf{#1}}}
\newcommand{\OtherTok}[1]{\textcolor[rgb]{0.56,0.35,0.01}{#1}}
\newcommand{\PreprocessorTok}[1]{\textcolor[rgb]{0.56,0.35,0.01}{\textit{#1}}}
\newcommand{\RegionMarkerTok}[1]{#1}
\newcommand{\SpecialCharTok}[1]{\textcolor[rgb]{0.00,0.00,0.00}{#1}}
\newcommand{\SpecialStringTok}[1]{\textcolor[rgb]{0.31,0.60,0.02}{#1}}
\newcommand{\StringTok}[1]{\textcolor[rgb]{0.31,0.60,0.02}{#1}}
\newcommand{\VariableTok}[1]{\textcolor[rgb]{0.00,0.00,0.00}{#1}}
\newcommand{\VerbatimStringTok}[1]{\textcolor[rgb]{0.31,0.60,0.02}{#1}}
\newcommand{\WarningTok}[1]{\textcolor[rgb]{0.56,0.35,0.01}{\textbf{\textit{#1}}}}
\usepackage{graphicx}
\makeatletter
\def\maxwidth{\ifdim\Gin@nat@width>\linewidth\linewidth\else\Gin@nat@width\fi}
\def\maxheight{\ifdim\Gin@nat@height>\textheight\textheight\else\Gin@nat@height\fi}
\makeatother
% Scale images if necessary, so that they will not overflow the page
% margins by default, and it is still possible to overwrite the defaults
% using explicit options in \includegraphics[width, height, ...]{}
\setkeys{Gin}{width=\maxwidth,height=\maxheight,keepaspectratio}
% Set default figure placement to htbp
\makeatletter
\def\fps@figure{htbp}
\makeatother
\setlength{\emergencystretch}{3em} % prevent overfull lines
\providecommand{\tightlist}{%
  \setlength{\itemsep}{0pt}\setlength{\parskip}{0pt}}
\setcounter{secnumdepth}{-\maxdimen} % remove section numbering
\ifluatex
  \usepackage{selnolig}  % disable illegal ligatures
\fi

\title{Cap 2. Exercises}
\author{}
\date{\vspace{-2.5em}}

\begin{document}
\maketitle

\hypertarget{conceptual}{%
\subsubsection{Conceptual}\label{conceptual}}

\begin{enumerate}
\def\labelenumi{\arabic{enumi}.}
\tightlist
\item
  For each of parts (a) through (d), indicate whether we would generally
  expect the performance of a flexible statistical learning method to be
  better or worse than an inflexible method. Justify your answer.
\end{enumerate}

\begin{enumerate}
\def\labelenumi{(\alph{enumi})}
\item
  The sample size n is extremely large, and the number of predictors p
  is small.
\item
  The number of predictors p is extremely large, and the number of
  observations n is small.
\item
  The relationship between the predictors and response is highly
  non-linear.
\item
  The variance of the error terms, i.e.~σ2 = Var(), is extremely high.
\end{enumerate}

\begin{enumerate}
\def\labelenumi{\arabic{enumi}.}
\setcounter{enumi}{1}
\tightlist
\item
  Explain whether each scenario is a classification or regression
  problem, and indicate whether we are most interested in inference or
  prediction. Finally, provide n and p.
\end{enumerate}

\begin{enumerate}
\def\labelenumi{(\alph{enumi})}
\item
  We collect a set of data on the top 500 firms in the US. For each firm
  we record profit, number of employees, industry and the CEO salary. We
  are interested in understanding which factors affect CEO salary.
\item
  We are considering launching a new product and wish to know whether it
  will be a success or a failure. We collect data on 20 similar products
  that were previously launched. For each product we have recorded
  whether it was a success or failure, price charged for the product,
  marketing budget, competition price, and ten other variables.
\item
  We are interesting in predicting the \% change in the US dollar in
  relation to the weekly changes in the world stock markets. Hence we
  collect weekly data for all of 2012. For each week we record the \%
  change in the dollar, the \% change in the US market, the \% change in
  the British market, and the \% change in the German market.
\end{enumerate}

\begin{enumerate}
\def\labelenumi{\arabic{enumi}.}
\setcounter{enumi}{2}
\tightlist
\item
  We now revisit the bias-variance decomposition.
\end{enumerate}

\begin{enumerate}
\def\labelenumi{(\alph{enumi})}
\item
  Provide a sketch of typical (squared) bias, variance, training error,
  test error, and Bayes (or irreducible) error curves, on a single plot,
  as we go from less flexible statistical learning methods towards more
  flexible approaches. The x-axis should representthe amount of
  flexibility in the method, and the y-axis should represent the values
  for each curve. There should be five curves. Make sure to label each
  one.
\item
  Explain why each of the five curves has the shape displayed in part
  (a).
\end{enumerate}

\begin{enumerate}
\def\labelenumi{\arabic{enumi}.}
\setcounter{enumi}{3}
\tightlist
\item
  You will now think of some real-life applications for statistical
  learning.
\end{enumerate}

\begin{enumerate}
\def\labelenumi{(\alph{enumi})}
\item
  Describe three real-life applications in which classification might be
  useful. Describe the response, as well as the predictors. Is the goal
  of each application inference or prediction? Explain your answer.
\item
  Describe three real-life applications in which regression might be
  useful. Describe the response, as well as the predictors. Is the goal
  of each application inference or prediction? Explain your answer.
\item
  Describe three real-life applications in which cluster analysis might
  be useful.
\end{enumerate}

\begin{enumerate}
\def\labelenumi{\arabic{enumi}.}
\setcounter{enumi}{4}
\item
  What are the advantages and disadvantages of a very flexible (versus a
  less flexible) approach for regression or classification? Under what
  circumstances might a more flexible approach be preferred to a less
  flexible approach? When might a less flexible approach be preferred?
\item
  Describe the differences between a parametric and a non-parametric
  statistical learning approach. What are the advantages of a parametric
  approach to regression or classification (as opposed to a
  nonparametric approach)? What are its disadvantages?
\end{enumerate}

\begin{Shaded}
\begin{Highlighting}[]
\FunctionTok{data}\NormalTok{(mtcars)}
\end{Highlighting}
\end{Shaded}


\end{document}
